\documentclass[]{article}
\usepackage{amsmath}
\usepackage{amsfonts}
\usepackage{amssymb}
\usepackage{tikz}
\usepackage{graphicx}
\usepackage{listings}

\definecolor{dkgreen}{rgb}{0,0.6,0}
\definecolor{gray}{rgb}{0.5,0.5,0.5}

\lstset{
  language=Python,
  breaklines=true,
  showstringspaces=false,
  frame=single,
  aboveskip=3mm,
  belowskip=3mm,
  columns=flexible,
  basicstyle={\small\ttfamily},
  numbers=none,
  numberstyle=\tiny\color{gray},
  keywordstyle=\color{blue},
  commentstyle=\color{gray},
  stringstyle=\color{dkgreen},
  breakatwhitespace=true,
  tabsize=3
}

\title{CAGD - Homework 6}
\author{Josefine St{\aa}l \& Erik Ackzell}

\begin{document}

\maketitle
\section*{Task 1}
In this task we plot a 3D Bretzel using B-splines. The code can be seen in Appendix I and the figure can be seen in the two figures below, from different angles.\\

\begin{figure}[h!]
	\includegraphics[scale=0.3]{bretzel3d2}
\end{figure}
\begin{figure}[h!]
	\includegraphics[scale=0.3]{bretzel3d}
\end{figure}
\newpage
\section*{Task 4}
In this task we plot a 3D cone using NURBS. The code can be seen in Appendix I and the figure can be seen below.\\

\begin{figure}[h!]
	\includegraphics[scale=0.3]{nurbscone}
\end{figure}
\newpage
\section*{Homework 4 task 4 do-over}
In this task we subdivide a 4th degree B-spline into its B\'{e}zier segments. This is done by first subdividing the B-spline at its internal knots into separate B-splines and then constructing B\'{e}zier segments using the control points of these separate B-splines.\\
The B\'{e}zier curves all have the the same degree as the original B-spline, namely 4. The original B-spline and its segments can be seen in the plot below and the code used to construct it can be seen in Appendix I.
\begin{figure}[h!]
	\includegraphics[scale=0.4]{task4_doover}
\end{figure}


\newpage
\section*{Appendix I}
\subsection*{Code for task 1}
\lstinputlisting[firstline=7]{task1.py}

\subsection*{Code for task 4}
\lstinputlisting[firstline=7]{task4.py}

\subsection*{Code for Homework 4 task 4 do-over}
\lstinputlisting[firstline=8]{homework4task4_doover.py}
\end{document}
